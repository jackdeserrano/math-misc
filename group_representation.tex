\documentclass [preview, border = 20pt] {standalone}



\usepackage {jack}
\linespread {1}


\begin {document}
\pagecolor{black}
\color{white}

\setlength{\parskip}{1 em}

Let $V$ be a vector space over a field $F$. Let $G$ be a group.

The \defn{general linear group} $\GL(V)$ of $V$ is the group of automorphisms (bijective linear transformations) of $V$.

A \defn{representation} of $G$ on $V$ is a group homomorphism
\begin{center}
\begin{tikzcd}
\rho :G\ar[r]&\GL(V)
\end{tikzcd}
\end{center}
and the vector space $V$. The \defn{dimension} of $(\rho, V)$ is
\[
\dim(\rho) := \dim(V).
\]

The representation $\rho$ is \defn{faithful} if and only if $\Ker(\rho)$ is the trivial group. 

A \defn{sub-representation} of a representation $(\rho, V)$ of $G$ is a representation $(\rho\big|_W, W)$ of $G$ where $W$ is a sub-space of $V$ and $(\rho\big|_W)(g) = \rho(g)\big |_W$.

A representation $(\rho, V)$ of $G$ is \defn{irreducible} if and only if it has precisely two sub-representations: $(\rho,(0))$ and $(\rho,V)$.

Suppose that $V$ is finite-dimensional over $F$. Let $(\rho, V)$ be a representation of $G$. The \defn{character} of $(\rho, V)$ is the function
\begin{center}
\begin{tikzcd}
\chi_\rho :G\ar[r]&F
\end{tikzcd}
\end{center}
given by
\[
\chi_\rho(g) := \Tr(\rho(g)).
\]
The character $\chi_\rho$ is \defn{irreducible} if and only if $(\rho, V)$ is irreducible. 

The \defn{degree} of the character $\chi_\rho$ is
\[
\deg(\chi_\rho) := \dim(\rho).
\]
Notice that, in characteristic zero,
\[
\deg(\chi_\rho) = \chi_\rho(1).
\]
Furthermore,
\begin{itemize}[topsep = 0 em, itemsep = 0 em]
\item $\chi_{\rho\oplus\sigma} = \chi_{\rho} + \chi_{\sigma}$;
\item $\chi_{\rho\otimes\sigma} = \chi_{\rho} \cdot \chi_{\sigma}$;
\item $\chi_{\rho^*} = \overline{\chi_{\rho}}$;
\end{itemize}
where $\rho^*$ is the conjugate transpose of $\rho$.

Add definitions of direct sum, tensor product, and conjugate transpose, Frobenius reciprocity, etc.
\end {document}
