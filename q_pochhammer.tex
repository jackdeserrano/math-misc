\documentclass [preview, border = 20pt] {standalone}

\usepackage {jack}
\linespread {1}

\begin {document}
\pagecolor{black}
\color{white}
The \defn{$q$-Pochhammer symbol} is
\[
 (a;q)_n := \prod_{k=0}^{n-1} (1-aq^k).
\]
Euler's function is 
\[
 \phi (q) = (q)_\infty := \prod_k (1-q^k).
\]
The formal power series expansion for $1/\phi(q)$ is 
\[
  \frac{1}{\phi(q)} = \sum_k p(k)q^k
\]
where $p$ is the partition function. Euler's pentagonal number theorem states that
\[
  \phi(q) = \sum_{n=-\infty}^\infty (-1)^n q^{(3n^2-n)/2}.
\]
Ramanujan discovered the remarkable identity
\[
 \phi(e^{-\pi}) = \frac{e^{\pi/24}\Gamma(1/4)}{2^{7/8}\pi^{3/4}}
\]
and others. Also,
\[
  \int_0^1 \phi(q)\, dq = \frac{8\sqrt{3/23} \pi\sinh(\sqrt{23}\pi/6)}{2\cosh(\sqrt{23}\pi/3)-1}.
\]
\end {document}
