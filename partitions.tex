\documentclass [preview, border = 20pt] {standalone}

\usepackage {jack}
\linespread {1}

\begin {document}
\pagecolor{black}
\color{white}
A \defn{partition} of a positive integer $n$ is a way of writing it as a sum of positive integers. The \defn{partition function} $p(n)$ counts the number of partitions of $n$. It has the generating function
\[
\sum_{n=0}^\infty p(n)q^n = \prod_{j=1}^\infty \sum_{i=0}^\infty q^{ji} = \prod_{j=1}^\infty (1-q^j)^{-1},
\]
but it has no closed form. Ramanujan discovered
\begin{align*}
p(5k+4) &\equiv 0\pmod 5;\\
p(7k+5) &\equiv 0\pmod 7;\\
p(11k+6) &\equiv 0\pmod {11}.
\end{align*}
These are \defn{Ramanujan's congruences}.
\end {document}