\documentclass [preview, border = 20pt] {standalone}

\usepackage {jack}
\linespread {1}

\begin {document}
\pagecolor{black}
\color{white}
Let $p$ and $q$ be two distinct odd prime numbers. Recall:
\[
\left( \frac q p\right) = \begin{cases}
  +1 & \textrm{$q$ is a quadratic residue modulo $p$}\\
  -1 & \textrm{otherwise.}
\end{cases}
\]
The law of quadratic reciprocity says that
\[
\left( \frac p q\right)\left( \frac q p\right) = (-1)^{(p-1)(q-1)/4}.
\]
Gauss first proved it. 

\footnotesize
See \ul{Disquisitiones arithmeticae} \S 4 arts. 107--50 and pp. 6--10 of Serre's \ul{A course in arithmetic}.
\end {document}
