\documentclass [preview, border = 20pt] {standalone}

\usepackage {jack}
\linespread {1}

\begin {document}
\pagecolor{black}
\color{white}
Let $n$ be an odd integer and define
\begin{align*}
\varepsilon(n) &\equiv \frac{n-1} 2 \pmod 2 = \begin{cases}
0& n\equiv +1\pmod 4\\
1&n\equiv -1 \pmod 4;
\end{cases}\\
\omega(n) &\equiv \frac{n^2-1} 8 \pmod 2 = \begin{cases}
0& n\equiv \pm 1\pmod 8\\
1&n\equiv \pm 5 \pmod 8.
\end{cases}
\end{align*}
Then
\begin{align*}
\left( \frac 1 p \right) &= 1;\\
\left( \frac{-1} p \right) &= (-1)^{\varepsilon(p)};\\
\left(\frac 2 p\right) &= (-1)^{\omega(p)}.
\end{align*}

Let $\alpha$ be a primitive $8$th root of unity in an algebraic closure $\Omega$ of $\F_p$. The element $y = \alpha + \alpha\inv$ verifies $y^2 =2$ (from $\alpha^4=-1$ it follows that $\alpha^2 + \alpha^{-2} = 0$). We have
\[
  y^p = \alpha^p + \alpha^{-p}.
\]
If $p\equiv \pm 1\pmod 8$, this implies $y^p=y$, thus
\[
  \left(\frac 2 p\right) = y^{p-1}=1.
\]
If $p\equiv \pm 5\pmod 8$, one finds $y^p = \alpha^5 +\alpha^{-5} = -(\alpha+\alpha\inv) = -y$. We deduce from this that $y^{p-1}=-1$, whence the third equality follows.
\end {document}