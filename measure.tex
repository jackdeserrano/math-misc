\documentclass [preview, border = 20pt] {standalone}

\usepackage {jack}
\newcommand{\hrulebar}{\\\hrule\text{}\\\footnotesize}
\linespread{1}
\DeclareMathOperator{\length}{length}

\begin {document}
\pagecolor{black}
\color{white}
Let $X$ be a set. A \defn{$\sigma$-algebra} over $X$ is a non-empty collection $\mathscr{S}$ of subsets of $X$ closed under complements, countable unions, and countable intersections. The pair $(X,\mathscr{S})$ is a \defn{measurable space}.

Let $(X, \mathscr{S})$ be a measurable space. Consider a function
\begin{center}
\begin{tikzcd}
\mu: \mathscr{S} \ar[r]& \R\cup \{-\infty,+\infty\}.
\end{tikzcd}
\end{center}
This function is a \defn{measure} if and only if
\begin{itemize}[label = ------, leftmargin = *]
\item for all $E\in \mathscr{S}$, $\mu(E)\ge 0$;
\item $\mu(\emptyset)=0$;
\item for all countable collections $(E_i)$ of pairwise disjoint sets in $\mathscr{S}$,
\[
  \mu\left(\bigcup_i E_i\right) = \sum_i\mu(E_i).
\]
\end{itemize}

Suppose that $E$ has finite measure. Then,
\[
 \mu(E) = \mu(E\cup\emptyset) = \mu(E) + \mu(\emptyset),
\]
so $\mu(\emptyset) = 0$.

The \defn{counting measure} is $\mu(E) := \# E$.

Write $\ell(I)$ for the length of an interval $I\subset\R$. Let $E\subset \R$. One defines the \defn{Lebesgue outer measure} by
\[
 \lambda^*(E) := \inf\left\{ \sum_j \ell(I_j) : \textrm{$(I_j)$ is a sequence of open intervals with $E\subset \bigcup_j I_j$}\right\}.
\]
A subset $E\subset \R$ satisfies the \defn{Carath\'eodory criterion} if and only if, for all $A\subset \R$,
\[
\lambda^*(A) = \lambda^*(A\cap E) + \lambda^*(A\cap E^\complement).
\]
The collection
\[
\mathscr{S} := \{ E\subset \R: \textrm{$E$ satisfies the {Carath\'eodory criterion}}\}
\]
is a $\sigma$-algebra over $\R$. The \defn{Lebesgue measure} $\lambda$ is $\lambda^*\big |_{\mathscr{S}}$. This construction extends naturally to $\R^n$. A function
\begin{center}
\begin{tikzcd}
f: X\ar[r]& \R
\end{tikzcd}
\end{center}
is \defn{measurable} if and only if $f\inv((\alpha,\infty)) \in \mathscr{S}$ for all $\alpha\in \R$.

\end {document}
