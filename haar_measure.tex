\documentclass [preview, border = 20pt] {standalone}

\usepackage {jack}
\newcommand{\hrulebar}{\\\hrule\text{}\\\footnotesize}
\linespread{1}
\DeclareMathOperator{\length}{length}

\begin {document}
\pagecolor{black}
\color{white}
A \defn{topological group} $G$ is a topological space and a group where the group operation and the inversion map are continuous. One views $G\times G$ with the product topology.

Let $(G,\cdot)$ be a locally compact Hausdorff topological group. The $\sigma$-algebra generated by all open subsets of $G$ is the \defn{Borel algebra} $\mathscr{B}(G)$. A member of the Borel algebra is a \defn{Borel set}.

A measure $\mu$ on the Borel algebra is \defn{left translation--invariant}, resp. \defn{right translation--invariant}, if and only if 
\[
  \mu(gS) = \mu(S),
\] 
resp. $\mu(Sg) = \mu(S)$, for all $g\in G$ and $S\in \mathscr{B}(G)$.

Haar's theorem says that there exists a unique non-trivial measure $\mu$ on $\mathscr{B}(G)$ that satisfies the following conditions:
\begin{itemize}[label = ------,leftmargin = *]
\item $\mu$ is left, resp. right, translation--invariant;
\item $\mu(K) < +\infty$ for all compact $K$;
\item $\mu$ is outer regular on Borel sets $S$:
\[
  \mu(S) = \inf_{\textrm{open $U\supset S$}} \mu(U);
\] 
\item $\mu$ is inner regular on open sets $U\subset G$:
\[
  \mu(U) = \sup_{\textrm{compact $K\subset U$}} \mu(K);
\]
\item $\mu(G) = 1$.
\end{itemize}
Such a measure is a left, resp. right, \defn{Haar measure}.
\end {document}