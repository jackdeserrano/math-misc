\documentclass [preview, border = 20pt] {standalone}

\usepackage {jack}
\linespread {1}

\begin {document}
\pagecolor{black}
\color{white}
Let $p$ be a prime number. For every $n\ge 1$, let $A_n = \Z/p^n\Z$; it is the ring of classes of integers mod $p^n$. An element of $A_n$ defines an element of $A_{n-1}$ in an obvious way. Thus, we obtain a homomorphism
\begin{center}
\begin{tikzcd}
\phi_n : A_n\ar[r]& A_{n-1}
\end{tikzcd}
\end{center} 
that is surjective and whose kernel is $p^{n-1}A_n$.

The sequence
\begin{center}
\begin{tikzcd}
\cdots \ar[r]& A_{n+1} \ar[r, "\phi_n"] & A_n\ar[r] &\cdots\ar[r] & A_2 \ar[r, "\phi_1"] & A_1
\end{tikzcd}
\end{center} 
forms a projective system indexed by the natural numbers.

The ring of $p$-adic integers $\Z_p$ is the projective limit of the system $(A_n,\phi_n)$ defined above.

By definition, an element of 
\[
\Z_p = \varprojlim (A_n,\phi_n)
\]
is a sequence $z = (\hdots, z_n,\hdots, z_1)$ with $z_n\in A_n$ and $\phi_n(z_n) = z_{n-1}$ if $n\ge 2$. Addition and multiplication in $\Z_p$ are defined ``coordinate by coordinate.'' That is, $\Z_p$ is a subring of the product $\prod_n A_n$. If we give $A_n$ the discrete topology and $\prod_n A_n$ the product topology, the ring $\Z_p$ inherits a topology that makes it a compact space, since it is closed in a product of compact spaces.
\end {document}