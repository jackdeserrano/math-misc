\documentclass [preview, border = 20pt] {standalone}



\usepackage {jack}
\linespread {1}

\let\Re\relax
\DeclareMathOperator{\Re}{Re}

\begin {document}
\pagecolor{black}
\color{white}

\setlength{\parskip}{1 em}

Recall that a function
\begin{center}
\begin{tikzcd}
\chi : \Z\ar[r] &\C
\end{tikzcd}
\end{center}
is a Dirichlet character (modulo $\mu$) if and only if
\begin{itemize}
\item $\chi(\alpha\beta) =\chi(\alpha)\chi(\beta)$;
\item $\chi(\alpha) = 0$ if and only if $\gcd(\alpha,\mu) >1$;
\item $\chi$ is $\mu$-periodic.
\end{itemize}


Given a Dirichlet character $\chi$, the associated Dirichlet $L$-function is the function
\[
L(s,\chi)  = \sum_{n=1}^\infty \frac{\chi(n)}{n^s}
\]
extended to $\C$ via analytic continuation.

\iffalse
Dirichlet $L$-functions have an Euler product 
\[
L(s,\chi) = \prod_{\textrm{prime numbers $p$}} (1- \chi(p)p^{-s})^{-1}
\]
for $\operatorname{Re}(s)>1$.
\fi

Suppose that $\chi$ is a primitive character modulo $q$. Write 
\[
\tau(\chi) = \sum_{n=1}^q \chi(n)\exp(2\pi i n/q).
\]
Also write
\[
a = \begin{cases}
 0 & \chi(-1)=1\\
 1 & \chi(-1)=-1
 \end{cases}
\]
and 
\[
\varepsilon(\chi) = \frac{\tau(\chi)}{i^a\sqrt q}.
\]
Then, $L(s,\chi)$ satisfies the functional equation
\[
L(s,\chi) = \varepsilon(\chi) 2^s\pi^{s-1}q^{1/2-s} \sin \left(\frac{\pi}{2}(s+a)\right)\Gamma(1-s)L(1-s,\overline{\chi}).
\]

More generally, given a sequence of complex numbers $(a_n)$, the associated $L$-series is the series
\[
\sum_{n=1}^\infty \frac{a_n}{n^s}.
\]
Then, one can determine whether this converges on some right half-plane of $\C$ and only thereafter determine whether it can be analytically continued to a meromorphic function on $\C$, which is the $L$-function associated to $(a_n)$.

Recall the definition for Ramanujan's $\tau$ function:
\[
\sum _{n=1}^\infty \tau(n) q^n = q\prod_{n=1}^\infty (1-q^n)^{24} = \Delta(z).
\]
This gives Ramanujan's $L$-function: for $\operatorname{Re}(s)>6$,
\[
L\sub{Ram}(s) = \sum_{n=1}^\infty \frac{\tau(n)}{n^s},
\]
and otherwise $L\sub{Ram}$ is defined by analytic continuation. It satisfies the functional equation
\[
L\sub{Ram}(s)\Gamma(s) (2\pi)^{-s} = L\sub{Ram}(12-s)\Gamma(12-s) (2\pi)^{-(12-s)}
\]
for almost all $s$ and has the Euler product
\[
L(s) = \prod_{\textrm{prime numbers}} \frac{1}{1-\tau(p)p^{-s} + p^{11-2s}}
\]
for $\operatorname{Re}(s)>7$.
Ramanujan conjectured that all non-trivial zeros of $L\sub{Ram}$ have real part $6$.
\end {document}
