\documentclass [preview, border = 20pt] {standalone}

\usepackage {jack}
\linespread {1}

\begin {document}
\pagecolor{black}
\color{white}
Let $A$ be a commutative ring. Let $\mathfrak a$ be an ideal of $A$. The \defn{radical} $\sqrt{\mathfrak a}$ of $\mathfrak a$ consists of the elements $\alpha\in A$ such that $\alpha^n \in \mathfrak a$ for some natural number $n$. The radical of the ideal $(0)$ is the \defn{nilradical} of $A$. We shall write
\[
\mathfrak N := \sqrt{(0)}.
\]
This is an ideal.

Proof. \quad
The element $0$ is nilpotent, so $0\in \mathfrak N$. Suppose $x,y\in \mathfrak N$. Say $x^m=y^n=0$. Then
\begin{align*}
(x-y)^{m+n} &= \sum_k \binom{m+n}{k}x^{k}y^{m+n-k}=0.
\end{align*}
So $x-y\in \mathfrak N$. Suppose $\alpha\in A$. Then 
\begin{align*}
(\alpha x)^m =\alpha^mx^m = 0.
\end{align*}
So $\alpha x\in \mathfrak N$.\hfill
$\square$

The ring $A$ is \defn{reduced} if and only if $\mathfrak N = (0)$.

The nilradical is the intersection of all prime ideals of $A$:
\[
\mathfrak N = \bigcap_{\mathfrak p\in \Spec(A)} \mathfrak p.
\]

\end {document}