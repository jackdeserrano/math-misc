\documentclass[preview, border=16pt]{standalone}

\usepackage [] {jack}
\newcommand{\hrulebar}{\\\hrule\text{}\\\footnotesize}
\linespread{1}

\begin{document}
\pagecolor{black}
\color{white}
Suppose $\alpha$ and $\beta$ are positive real numbers. Define $\alpha_0 := \alpha$ and $\beta_0=\beta$. Inductively, define
\begin{align*}
\alpha_{n+1} &:= \frac{1}{2}(\alpha_n + \beta_n);\\
\beta_{n+1} &:= \sqrt{\alpha_n\beta_n}.
\end{align*}
The induced sequences $(\alpha_n)$ and $(\beta_n)$ converge to the \defn{arithmetic-geometric mean} of $\alpha$ and $\beta$, which one notates $\agm (\alpha,\beta)$.

Theorem.\quad
If $\alpha\ge\beta>0$, then
\[
\agm (\alpha,\beta) \int_0^{\pi/2} (\alpha^2\cos^2(\phi) + \beta^2\sin^2(\phi))^{-1/2}\,d\phi = \frac{\pi}{2}.
\]
Proof.\quad
Define
\[
\II(\alpha,\beta) := \int_0^{\pi/2} (\alpha^2\cos^2(\phi) + \beta^2\sin^2(\phi))^{-1/2}\,d\phi
\]
and define $\tilde\phi$ such that
\[
\sin (\phi) = \frac{2\alpha\sin( \tilde \phi)}{\alpha+\beta+(\alpha-\beta)\sin^2(\tilde\phi)}.
\]
Notice that
\begin{align*}
\cos (\phi) &= \frac{2\cos(\tilde \phi)(\alpha_1^2 \cos^2(\tilde\phi) + \beta_1^2 \sin^2(\tilde \phi))^{1/2}}{\alpha+\beta+(\alpha-\beta)\sin^2(\tilde \phi)};\\
(\alpha^2\cos^2(\phi) + \beta^2\sin^2(\phi))^{1/2} &= \alpha\frac{\alpha+\beta-(\alpha-\beta)\sin^2(\tilde\phi)}{\alpha+\beta+(\alpha-\beta)\sin^2(\tilde\phi)}.
\end{align*}
Therefore,
\[
(\alpha^2\cos^2(\phi) + \beta^2\sin^2(\phi))^{-1/2}\,d\phi = (\alpha_1^2\cos^2(\tilde\phi) + \beta_1^2 \sin^2(\tilde\phi))^{-1/2}\, d\tilde \phi,
\]
whence $\II(\alpha,\beta) = \II(\alpha_1,\beta_1)$ follows.
Now, one sees that
\[
\II(\alpha,\beta) = \II(\alpha_1,\beta_1) = \cdots,
\]
so
\[
\II(\alpha,\beta) = \lim_{n\to\infty} \II(\alpha_n,\beta_n) = \frac{\pi}{2\agm (\alpha,\beta)}
\]
since $(\alpha_n^2\cos^2(\phi) + \beta_n^2\sin^2(\phi))^{-1/2}$ converges uniformly to $\agm (\alpha,\beta)\inv$.\hfill 
$\square$

Gauss proved that 
\[
\agm (1,\sqrt{2}) = \frac{\pi}{\varpi} \approx 1.19814
\]
where $\varpi$ is the \defn{lemniscate constant}:
\[
\varpi := 2\int_{0}^1\frac{ds}{\sqrt{1-s^4}}\approx 2.62206.
\]

\footnotesize
See \underline{Werke} (Gauss) III p. 352 and \underline{Gesammelte Werke} (Jacobi) I p. 152.
\end{document}