\documentclass [preview, border = 20pt] {standalone}

\usepackage {jack}
\linespread {1}

\begin {document}
\pagecolor{black}
\color{white}
Euler's totient function $\varphi(n)$ counts the number of positive integers less than $n$ that are prime to $n$. One has
\[
\varphi(n) = n \prod_{p\,\mid\,n} \left( 1-\frac 1 p\right).
\]
Gauss found that this multiplicative function satisfies
\[
\sum_{d\,\mid\,n}\varphi(d) = n.
\]
Euler's theorem, a generalization of Fermat's little theorem, states that, if $a$ and $n$ are relatively prime, then
\[
a^{\varphi(n)} \equiv 1\pmod n.
\]

\footnotesize
See \ul{Disquisitiones arithmeticae} art. 39 p. 21.
\end {document}
